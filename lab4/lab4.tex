\documentclass[12pt]{article}
\usepackage[margin=1in]{geometry}
\usepackage{graphicx}
\usepackage{amsmath, amssymb}
\usepackage{booktabs}
\usepackage{float}
\usepackage{caption}
\captionsetup[figure]{labelfont=bf, font=it}
\usepackage{hyperref}
\usepackage{setspace}
\usepackage{siunitx}

\hypersetup{colorlinks=true, linkcolor=blue, urlcolor=blue, citecolor=blue}

\title{\textbf{\LARGE Standing Waves on a String} \\ \large Measuring Wave Speed - Lab 4}
\author{Sean Rockwitz, Brandon Salinas, Erick Puli, Abhiraj Singh}
\date{\textit{11 February, 2026} \\ \textit{PHYS 180, Dr.\ Ray Sameshima} \\ \textit{New York Institute of Technology}}

\begin{document}

% ============================================================
% TITLE PAGE
% ============================================================
\begin{titlepage}
    \centering
    \vspace*{1.2cm}
    {\Huge \textbf{Standing Waves on a String} \par}
    \vspace{0.3cm}
    {\Large Measuring Wave Speed - Lab 4 \par}
    \vspace{0.6cm}
    \rule{\textwidth}{0.4pt}
    \vspace{0.5cm}

    {\large Sean Rockwitz, Brandon Salinas, Erick Puli, Abhiraj Singh \par}
    \vspace{0.2cm}
    {\large \textit{11 February, 2026} \par}
    {\large \textit{PHYS 180, Dr.\ Ray Sameshima} \par}
    {\large \textit{New York Institute of Technology} \par}

    \vspace{0.8cm}

    \begin{figure}[H]
        \centering
        \includegraphics[width=0.7\textwidth]{IMG_3431.jpg}
        \caption*{\textbf{Figure 1.} \textit{Standing wave theory and speed formulae on the board during lab.}}
    \end{figure}

\end{titlepage}

% ---- 1. LEARNING OBJECTIVES ----
\section{Learning Objectives}

The purpose of this lab was to measure wave speed along a string by creating standing waves at different frequencies using a sine wave generator. We worked with two strings (a thin green one and a thicker orange one), driving each at a range of frequencies to identify the resonant modes---the frequencies where the string settles into a clean, stable pattern with well-defined nodes and antinodes. From these resonant frequencies, we computed the wave speed and compared it against the theoretical prediction $V = \sqrt{F_T/\mu}$, which comes from the string's tension and linear mass density.

We split the work as follows: I (Sean) counted nodes along the string, Abhiraj operated the sine wave generator and dialed in frequencies, Erick recorded all frequency and mode data in his spreadsheet, and Brandon assisted with setup and verification throughout.

% ---- 2. MAIN CONCEPTS ----
\section{Main Concepts and Formulae}

Standing waves form on a string fixed at both ends when it is moved at the right frequency. The forward wave reflects off the far end and combines with the incoming wave, and at certain frequencies the two line up to produce nodes (points of zero displacement) and antinodes (points of maximum displacement). These special frequencies are the resonant frequencies.

\subsection{Wave Speed on a String}

The speed of a transverse wave on a string depends on the tension $F_T$ and the linear mass density $\mu = m/L$. The wave equation governing this is

\begin{equation}
    \mu \frac{\partial^2 y}{\partial t^2} = F_T \frac{\partial^2 y}{\partial x^2}
\end{equation}

from which the wave speed follows:

\begin{equation}
    |v|^2 = \frac{F_T}{\mu} \quad \implies \quad |v| = \sqrt{\frac{F_T}{\mu}}
\end{equation}

In our setup, the tension comes from a mass hanging off the string, so $F_T = mg$.

\subsection{Resonant Frequencies and Standing Wave Modes}

For the $n$-th mode, $n$ half-wavelengths fit within the vibrating length $L$, giving

\begin{equation}
    \lambda_n = \frac{2L}{n}, \quad n = 1, 2, 3, \ldots
\end{equation}

Since we know speed equals frequency times wavelength, the wave speed for each mode is

\begin{equation}
    |v_n| = f_n \cdot \lambda_n = f_n \cdot \frac{2L}{n}
\end{equation}

This value should be the same regardless of mode number, since the wave speed is a property of the string, not the frequency. As a result, plotting $f_n$ vs.\ $n$ should yield a straight line whose slope equals $V/(2L)$.

% ---- 3. PROCEDURE ----
\section{Procedure}

The setup was straightforward. One end of the string attaches to the sine wave generator, which shakes it back and forth. The string runs across the table, over a pulley, and a hanging mass on the other side provides the tension. The effective vibrating length---from the generator to the pulley---was $L_{\text{eff}} = 2.352$ m for both strings.

We tested two strings with different properties:

\textbf{Green string:} Total length 2.982 m, string mass 0.00280 kg, hanging mass 0.500 kg ($F_T = 4.91$ N). This was the lighter of the two.

\textbf{Orange string:} Total length 3.01 m, string mass 0.01803 kg, hanging mass 1.000 kg ($F_T = 9.81$ N). Heavier string with more tension.

For each string, Abhiraj gradually increased the generator frequency until a clean standing wave pattern appeared. Resonance is fairly easy to identify visually---the string transitions from vibrating somewhat randomly to settling into a well-defined pattern with clear, stationary nodes. I counted the nodes while Erick recorded the frequency $f_n$ and mode number $n$. We repeated this for as many modes as we could reliably identify ($n = 2$ through $n = 7$ or $8$), and cross-checked our values against the class data on the board. The class average for the green string's linear density was $\mu_{\text{avg}} = 9.52 \times 10^{-4}$ kg/m.

% ---- 4. EXPERIMENTAL DATA ----
\section{Experimental Data and Graphical Analysis}

\subsection{Green String}

\begin{table}[H]
    \centering
    \caption{Resonant frequency and wave speed data for the green string. Hanging mass $= 0.500$ kg, $L_{\text{eff}} = 2.352$ m.}
    \label{tab:green}
    \begin{tabular}{cccc}
        \toprule
        $n$ (nodes) & $f_n$ (Hz) & $\lambda_n$ (m) & $V_n$ (m/s) \\
        \midrule
        2 & 33.6 & 2.352 & 79.0 \\
        3 & 50.6 & 1.568 & 79.3 \\
        4 & 62.5 & 1.176 & 73.5 \\
        5 & 78.7 & 0.9408 & 74.0 \\
        6 & 94.7 & 0.7840 & 74.2 \\
        7 & 124.4 & 0.6720 & 83.6 \\
        \bottomrule
    \end{tabular}
\end{table}

The wave speed values show some scatter across modes. The $n = 7$ reading came in noticeably high at 83.6 m/s, likely because at higher modes the nodes are packed closer together and it becomes harder to determine the exact resonant frequency. This is discussed further in the error analysis.

\begin{figure}[H]
    \centering
    \includegraphics[width=0.5\textwidth]{fig1_green_fn.pdf}
    \caption{Green string: resonant frequency $f_n$ vs.\ number of nodes $n$. Linear fit slope $= 17.21$ Hz/node.}
    \label{fig:green_fn}
\end{figure}

\begin{figure}[H]
    \centering
    \includegraphics[width=0.5\textwidth]{fig2_green_vn.pdf}
    \caption{Green string: wave speed $V_n$ vs.\ number of nodes $n$. Dashed: average (77.3 m/s); dotted: theory (72.3 m/s).}
    \label{fig:green_vn}
\end{figure}

\subsection{Orange String}

\begin{table}[H]
    \centering
    \caption{Resonant frequency and wave speed data for the orange string. Hanging mass $= 1.000$ kg, $L_{\text{eff}} = 2.352$ m.}
    \label{tab:orange}
    \begin{tabular}{cccc}
        \toprule
        $n$ (nodes) & $f_n$ (Hz) & $\lambda_n$ (m) & $V_n$ (m/s) \\
        \midrule
        2 & 17.4 & 2.352 & 40.9 \\
        3 & 26.5 & 1.568 & 41.6 \\
        4 & 37.4 & 1.176 & 44.0 \\
        5 & 47.6 & 0.9408 & 44.8 \\
        6 & 53.6 & 0.7840 & 42.0 \\
        8 & 71.6 & 0.5880 & 42.1 \\
        \bottomrule
    \end{tabular}
\end{table}

The orange string data is noticeably tighter than the green. This likely comes down to the heavier string producing more pronounced standing wave patterns under higher tension, making the resonant frequencies easier to identify with confidence. During the experiment, the transition into resonance on the orange string was more abrupt---the amplitude would increase sharply and the nodes would become very well-defined.

\begin{figure}[H]
    \centering
    \includegraphics[width=0.5\textwidth]{fig3_orange_fn.pdf}
    \caption{Orange string: resonant frequency $f_n$ vs.\ number of nodes $n$. Linear fit slope $= 9.02$ Hz/node.}
    \label{fig:orange_fn}
\end{figure}

\begin{figure}[H]
    \centering
    \includegraphics[width=0.5\textwidth]{fig4_orange_vn.pdf}
    \caption{Orange string: wave speed $V_n$ vs.\ number of nodes $n$. Dashed: average (42.6 m/s); dotted: theory (40.5 m/s).}
    \label{fig:orange_vn}
\end{figure}

\subsection{Combined Overview}

\begin{figure}[H]
    \centering
    \includegraphics[width=0.75\textwidth]{fig5_combined.pdf}
    \caption{Combined 2$\times$2 plot: frequency and wave speed results for both strings.}
    \label{fig:combined}
\end{figure}

% ---- SUMMARY TABLE ----
\subsection{Summary of Results}

Table~\ref{tab:summary} compares the wave speed obtained from standing wave modes (Method 1) with the theoretical value from $V = \sqrt{F_T/\mu}$ (Method 2).

\begin{table}[H]
    \centering
    \caption{Wave speed comparison (m/s). Method 1: from modes; Method 2: $V = \sqrt{F_T/\mu}$.}
    \label{tab:summary}
    \begin{tabular}{@{} l c c c c @{}}
        \toprule
        \textbf{String} & \textbf{Method 1 (m/s)} & \textbf{Method 2 (m/s)} & \textbf{$\Delta V$ (m/s)} & \textbf{\% diff.} \\
        \midrule
        Green  & $77.3 \pm 3.5$ & $72.3 \pm 0.1$ & $5.0 \pm 3.5$ & $6.9 \pm 4.8$ \\
        Orange & $42.6 \pm 1.2$ & $40.5 \pm 0.1$ & $2.1 \pm 1.2$ & $5.2 \pm 3.0$ \\
        \bottomrule
    \end{tabular}
\end{table}

\begin{table}[H]
    \centering
    \caption{Input parameters for each string.}
    \label{tab:summary_params}
    \begin{tabular}{@{} l c c @{}}
        \toprule
        \textbf{Parameter} & \textbf{Green} & \textbf{Orange} \\
        \midrule
        $m$ (kg) & 0.00280 & 0.01803 \\
        $L$ (m) & 2.982 & 3.01 \\
        $L_{\text{eff}}$ (m) & 2.352 & 2.352 \\
        $m_h$ (kg) & 0.500 & 1.000 \\
        $\mu$ (kg/m) & $(9.39 \pm 0.03) \times 10^{-4}$ & $(5.99 \pm 0.02) \times 10^{-3}$ \\
        $F_T$ (N) & $4.91 \pm 0.01$ & $9.81 \pm 0.01$ \\
        \bottomrule
    \end{tabular}
\end{table}

% ---- 5. CALCULATION AND ERROR ANALYSIS ----
\section{Calculation and Error Analysis}

Uncertainties are propagated following the guidelines in NIST SP 811 \cite{nist811}. Raw data is in Tables~\ref{tab:green} and~\ref{tab:orange}.

\subsection{Method 1: Wave Speed from Standing Wave Modes}

For each resonant mode $n$, the wavelength is determined by the vibrating length:
\begin{equation}
    \lambda_n = \frac{2L_{\text{eff}}}{n}, \quad L_{\text{eff}} = 2.352\;\text{m}.
\end{equation}
The wave speed for that mode then follows from
\begin{equation}
    |v_n| = f_n \cdot \lambda_n = f_n \cdot \frac{2L_{\text{eff}}}{n}.
\end{equation}
We measured six modes for each string and averaged the resulting speeds:
\begin{equation}
    \overline{V}_{\text{exp}} = \frac{1}{N} \sum_{i=1}^{N} V_i.
\end{equation}
The uncertainty comes from the standard deviation across those six values:
\begin{equation}
    \sigma_V = \sqrt{\frac{1}{N-1} \sum_{i=1}^{N} (V_i - \overline{V})^2}.
\end{equation}
This gives $\overline{V}_{\text{green}} = 77.3 \pm 3.5$ m/s and $\overline{V}_{\text{orange}} = 42.6 \pm 1.2$ m/s. The larger uncertainty on the green string reflects the greater scatter in its mode-by-mode speed values.

\subsection{Method 2: Wave Speed from Tension and Linear Density}

The theoretical speed uses only the physical properties of the string. The tension equals the weight of the hanging mass:
\begin{equation}
    F_T = m_h g.
\end{equation}
The linear mass density is
\begin{equation}
    \mu = \frac{m_{\text{string}}}{L_{\text{string}}}.
\end{equation}
Substituting into the wave speed formula:
\begin{equation}
    V_{\text{theory}} = \sqrt{\frac{F_T}{\mu}}.
\end{equation}
After propagating uncertainties: $V_{\text{theory,green}} = 72.3 \pm 0.1$ m/s and $V_{\text{theory,orange}} = 40.5 \pm 0.1$ m/s. The theoretical uncertainties are small because mass and length can be measured with good precision.

\subsection{Comparison of the Two Methods}

The difference between methods is
\begin{equation}
    \Delta V = \left| \overline{V}_{\text{exp}} - V_{\text{theory}} \right|.
\end{equation}
with uncertainty
\begin{equation}
    \delta(\Delta V) = \sqrt{(\delta\overline{V}_{\text{exp}})^2 + (\delta V_{\text{theory}})^2}.
\end{equation}
The percent difference is
\begin{equation}
    \%\,\text{diff} = \frac{\Delta V}{V_{\text{theory}}} \times 100\%.
\end{equation}
For the green string: $\Delta V = 5.0 \pm 3.5$ m/s ($6.9 \pm 4.8\%$). For the orange: $\Delta V = 2.1 \pm 1.2$ m/s ($5.2 \pm 3.0\%$). Both results agree within their uncertainty ranges, which supports the validity of the wave speed formula.

\subsection{Sources of Error}

The biggest problem in Method 1 was figuring out when we had actually found the resonant frequency. The generator dial is continuous, so you're slowly turning it and watching the string until the pattern looks the cleanest. There's definitely a judgment call involved, and at higher modes where the nodes are packed closer together, the differences between nearby frequencies get more subtle. This is probably what caused our $n = 7$ green string reading to be so far off from the rest.

The effective length is another source of uncertainty. We measured $L_{\text{eff}}$ from the generator to the pulley, but neither end is really a perfect node. The generator end moves a little (that's how it drives the string in the first place), and the pulley has a finite width to it. So the actual vibrating length could be off by a few millimeters from what we used in the calculations.

The orange string gave tighter results throughout, which we think comes down to it being heavier with more tension. The standing wave patterns on the orange string were just more visible and more defined, so it was easier to tell when we had actually hit resonance. With the green string, especially above $n = 5$ or so, it was sometimes hard to tell if we were on the right frequency or just close to it.

% ============================================================
% CONCLUSION
% ============================================================
\section*{Conclusion}

Both strings produced wave speeds that agreed with the theoretical prediction from $V = \sqrt{F_T/\mu}$ within uncertainty---approximately 6.9\% difference for the green string and 5.2\% for the orange. The $f_n$ vs.\ $n$ plots were linear as expected, and the wave speed showed no systematic trend with mode number, consistent with a constant wave speed across all modes.

The orange string gave tighter results throughout, which we believe is due to the heavier string and higher tension producing more distinct standing wave patterns. The main limitation of our approach was the subjective tuning of the resonant frequency; a sensor-based method for detecting peak amplitude would likely reduce the scatter significantly. That said, the agreement between our two methods is strong enough to confirm that the wave equation correctly predicts the speed of transverse waves on a string.

\vspace{0.5cm}
\noindent LaTeX source: \href{https://github.com/SRock44/physics-latex}{github.com/SRock44/physics-latex} (file \texttt{lab4/lab4.tex}).

\begin{thebibliography}{9}
\bibitem{nist811}
Taylor, B. N., and Kuyatt, C. E. (2008). \textit{Guidelines for Evaluating and Expressing the Uncertainty of NIST Measurement Results}. NIST Special Publication 811, 2008 Edition. National Institute of Standards and Technology, Gaithersburg, MD. Available at: \url{https://nvlpubs.nist.gov/nistpubs/Legacy/SP/nistspecialpublication811e2008.pdf}
\end{thebibliography}

\end{document}